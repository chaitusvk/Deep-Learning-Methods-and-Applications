\chapter{Detailed design}
\label{Chapter5}
\section{Module Detailed design}

\subsection{Main }
This function is created by the QT creator. It contains the QApplication object and the Mainwindow objects. This function shows the main GUI page. 
\subsubsection{Attributes}
Nil
\subsubsection{Member Functions }
Nil
\subsubsection{Slots}
Nil
\subsubsection{Signals}
Nil
\subsubsection{Internal design details }
Nil

\subsection{MainWindow}
\subsubsection{Attributes}
\begin{enumerate}
\item [$\rhd$] class NetworkTx;
\item [$\rhd$] class irig;
\item [$\rhd$] class IRIG$\_$THREAD;
\item [$\rhd$] class LogData;
\item [$\rhd$]QThread NetworkTxThread1;
\item [$\rhd$]QThread NetworkTxThread2;
\item [$\rhd$]QThread LogThread;
\item [$\rhd$]NetworkTx* m$\_$NetTx;
\item [$\rhd$] NetworkTxThread* m$\_$NetTx1;
\item [$\rhd$] NetworkTxThread* m$\_$NetTx2;
\item [$\rhd$] LogData* m$\_$DLog;
\item [$\rhd$] 
\item [$\rhd$] irig* m$\_$irig;
\item [$\rhd$] IRIG$\_$THREAD* m$\_$irig1;
\item [$\rhd$] IRIG$\_$THREAD* m$\_$irig2;
\item [$\rhd$] IRIG$\_$THREAD* m$\_$irig3;
\item [$\rhd$] IRIG$\_$THREAD* m$\_$irig4;
\item [$\rhd$] IRIG$\_$THREAD* m$\_$irig4;
\item [$\rhd$] QTimer *Timer10Hz;
\item [$\rhd$] Ui::MainWindow *ui;
\item [$\rhd$] QTimer *timer;
\item [$\rhd$] int ret;
\item [$\rhd$] int Hmsecs$\_$Count;
\item [$\rhd$] int SecondsCount;
\item [$\rhd$] quint8 LogDataFlag;
\item [$\rhd$] quint8 LogOverallFlag;
\item [$\rhd$] int irigStatus[8];

\end{enumerate}

\subsubsection{Member Functions }
\begin{enumerate}
	\item [$\blacklozenge$] void ExitOnError(void);
	\item [$\blacklozenge$] void InitializeIrigCards(void);
	\item [$\blacklozenge$] void DisScreen(void);
	\item [$\blacklozenge$] int  OpenFiles(void);
	
\end{enumerate}

\subsubsection{Slots}
\begin{enumerate}
	\item [$\blacklozenge$] void on$\_$APP$\_$EXIT$\_$clicked();
	\item [$\blacklozenge$] void on$\_$LOGenable$\_$clicked();
	\item [$\blacklozenge$] void on$\_$LOGoverall$\_$clicked();
	\item [$\blacklozenge$] void on$\_$irig1DataRxd(int NoOfBytes);
	\item [$\blacklozenge$] void on$\_$irig2DataRxd(int NoOfBytes);
	\item [$\blacklozenge$] void on$\_$irig3DataRxd(int NoOfBytes);
	\item [$\blacklozenge$] void on$\_$irig4DataRxd(int NoOfBytes);
	\item [$\blacklozenge$] void on$\_$irig4DataRxd(int NoOfBytes);
	\item [$\blacklozenge$] void TimerTicked();
	
	
\end{enumerate}

\subsubsection{Signals}
\begin{enumerate}
	\item [$\blacklozenge$] void ExitOnError(void);

	
\end{enumerate}

\subsubsection{Internal design details }
\begin{enumerate}
	\item  \textbf{MainWindow::MainWindow(QWidget *parent) :
    QMainWindow(parent),
    ui(new Ui::MainWindow)}
	\begin{enumerate}
		\item \textit{Inputs :} Nil.
		\item \textit{Functional details:} The constructor is derived from the QObject class.
		\item \textit{Outputs:} Nil.
		\item \textit{Implementation details:} In this function all IRIG threads and network threads are initilised by reading config files and made to run 
		\item \textit{File operations:} Nil.
	\end{enumerate}
	
	\item  \textbf{MainWindow::$ \sim $MainWindow()}
	\begin{enumerate}
		\item \textit{Inputs :} Nil
		\item \textit{Functional details:} 
		\item \textit{Outputs:} Nil
		\item \textit{Implementation details:} This function is Desctructor for main window class.all threads are stopped and deleted 
		\item \textit{File operations:} Nil.
	\end{enumerate}
	
	\item  \textbf{void MainWindow::ExitOnError(void)}
	\begin{enumerate}
		\item \textit{Inputs :} Nil
		\item \textit{Functional details:} This function is called to exit the program . this function exits all running thread ,delete thier instances  and also deleted all pointers to avoid memory leaks
		\item \textit{Outputs:} NIl
		\item \textit{Implementation details:} 
		\item \textit{File operations:} Nil
	\end{enumerate}
	
	\item  \textbf{int MainWindow::OpenFiles(void)}
	\begin{enumerate}
		\item \textit{Inputs :} Nil 
		\item \textit{Functional details:} 
		\item \textit{Outputs:} file opened status
		\item \textit{Implementation details:}  This function opens the config files
		\item \textit{File operations:} opens config file namely "irigcards.txt" "TxSocket".
	\end{enumerate}
	
	\item  \textbf{void MainWindow::on$\_$irig1DataRxd(int NoOfBytes)}
	\begin{enumerate}
		\item \textit{Inputs :} No of Bytes of Time code data received from card
		\item \textit{Functional details:} 
		\item \textit{Outputs:} read status 
		\item \textit{Implementation details:} This function takes the time data and logs UT-1 time
		\item \textit{File operations:} Nil
	\end{enumerate}
	
	\item  \textbf{void MainWindow::on$\_$irig2DataRxd(int NoOfBytes)}
	\begin{enumerate}
		\item \textit{Inputs :} No of Bytes of Time code data received from card
		\item \textit{Functional details:} 
		\item \textit{Outputs:} read status 
		\item \textit{Implementation details:} This function takes the time data and logs UT-2 time
		\item \textit{File operations:} Nil
	\end{enumerate}
	
	\item  \textbf{void MainWindow::on$\_$irig3DataRxd(int NoOfBytes)}
	\begin{enumerate}
		\item \textit{Inputs :} No of Bytes of Time code data received from card
		\item \textit{Functional details:} 
		\item \textit{Outputs:} read status 
		\item \textit{Implementation details:} This function takes the time data and logs CDT-1 time
		\item \textit{File operations:} Nil
	\end{enumerate}
	
	\item  \textbf{void MainWindow::on$\_$irig4DataRxd(int NoOfBytes)}
	\begin{enumerate}
		\item \textit{Inputs :} No of Bytes of Time code data received from card
		\item \textit{Functional details:} 
		\item \textit{Outputs:} read status 
		\item \textit{Implementation details:} This function takes the time data and logs CDT-2 time
		\item \textit{File operations:} Nil
	\end{enumerate}
	
	
	\item  \textbf{void MainWindow::on$\_$LOGenable$\_$clicked()}
	\begin{enumerate}
		\item \textit{Inputs :} Nil
		\item \textit{Functional details:} 
		\item \textit{Outputs:} Nil
		\item \textit{Implementation details:} This slot function called when LOG button pressed in gui.this function enables the log.
		\item \textit{File operations:} Nil
	\end{enumerate}
	
	

	
\end{enumerate}


\subsection{irig}
\subsubsection{Attributes}
\begin{enumerate}
\item [$\rhd$]struct IRIG$\_$NODE;
\item [$\rhd$]QList$<$IRIG$\_$NODE*$>$ irig$\_$list;
\item [$\rhd$] uint NoOfIrigCards;
\item [$\rhd$] uint InterruptRate;
\item [$\rhd$] int irigb$\_$size;

\end{enumerate}

\subsubsection{Member Functions }
\begin{enumerate}
	\item [$\blacklozenge$] irig::irig(QObject *parent)
	\item [$\blacklozenge$] irig::irig()
	\item [$\blacklozenge$] int irig::Read$\_$IRIG$\_$File(QString file$\_$name)
\end{enumerate}

\subsubsection{Slots}
Nil
\subsubsection{Signals}
Nil
\subsubsection{Internal design details }
\begin{enumerate}
	\item  \textbf{irig::irig(QObject *parent)}
	\begin{enumerate}
		\item \textit{Inputs :} Nil.
		\item \textit{Functional details:} The constructor is derived from the QObject class.
		\item \textit{Outputs:} Nil.
		\item \textit{Implementation details:} It is a constructor for irig class.
		\item \textit{File operations:} Nil.
	\end{enumerate}
	
	\item  \textbf{irig::irig()}
	\begin{enumerate}
		\item \textit{Inputs :} Nil
		\item \textit{Functional details:} 
		\item \textit{Outputs:} Nil
		\item \textit{Implementation details:} This function is Desctructor for irig class.
		\item \textit{File operations:} Nil.
	\end{enumerate}
	
	\item  \textbf{int irig::Read$\_$IRIG$\_$File(QString file$\_$name)}
	\begin{enumerate}
		\item \textit{Inputs :} File name of irig configuraion file //"irigcards.txt"
		\item \textit{Functional details:} 
		\item \textit{Outputs:} file read status 
		\item \textit{Implementation details:} This function reads the irigcards configuration file and initialises the irig node class.
		\item \textit{File operations:} opens "irigcards.txt" file and closes it after reading.
	\end{enumerate}
	

	
\end{enumerate}



\subsection{IRIG$\_$THREAD}
\subsubsection{Attributes}
\begin{enumerate}
	\item [$\rhd$] int fp;
	\item [$\rhd$] int fd[6]; 
	\item [$\rhd$] int NoOfCards;
	\item [$\rhd$] quint8 UT1[10],UT2[10], CDT1[10],CDT2[10],CDT3[10];
	\item [$\rhd$] int gDevNo;;
	\item [$\rhd$] int sfd[6];
	\item [$\rhd$] sigset$\_$t mask;
	\item [$\rhd$] int modevalue[8];
\end{enumerate}


\subsubsection{Member Functions }
\begin{enumerate}
	\item [$\blacklozenge$] int ConfigureIrigCard(int CardNo, QList$<$IRIG$\_$NODE*$>$ irig$\_$list, uint NoOfIrigCards,uint InterruptRate);	
	\item [$\blacklozenge$] void SelectDevice(int CardNo);
	\item [$\blacklozenge$] void run();
	\item [$\blacklozenge$] void read1();

	
\end{enumerate}
\subsubsection{Slots}
Nil


\subsubsection{Signals}
\begin{enumerate}

	\item [$\blacklozenge$] void Irig$\_$Rxd$\_$Flag(int NoOfBytes);

\end{enumerate}
\subsubsection{Internal design details }
\begin{enumerate}
	\item  \textbf{IRIG$\_$THREAD::IRIG$\_$THREAD()}
	\begin{enumerate}
		\item \textit{Inputs :} Nil.
		\item \textit{Functional details:} The constructor is derived from the QObject class.it initialises various variables in the class 
		\item \textit{Outputs:} Connects signals to slots.
		\item \textit{Implementation details:} Qt Creator 5.4, C++ language.
		\item \textit{File operations:} Nil.
	\end{enumerate}

\item  \textbf{IRIG$\_$THREAD::$\sim$ IRIG$\_$THREAD()}
\begin{enumerate}
	\item \textit{Inputs :} Nil
	\item \textit{Functional details:} This function is Destructor for the class
	\item \textit{Outputs:} Nil.
	\item \textit{Implementation details:} Qt Creator 5.4, C++ language.
	\item \textit{File operations:} Nil.
\end{enumerate}
\item  \textbf{int IRIG$\_$THREAD::ConfigureIrigCard(int CardNo,QList$<$IRIG$\_$NODE*$>$ irig$\_$list, uint NoOfIrigCards, uint InterruptRate)}
\begin{enumerate}
	\item \textit{Inputs :} Cardno, irigcard list,numofcards interruptrate
	\item \textit{Functional details:} This function opens the device and configures the device with parameters read from config file ie,"irigcard.txt"
	\item \textit{Outputs:} Nil.
	\item \textit{Implementation details:} Qt Creator 5.4, C++ language.
	\item \textit{File operations:} Opens the irig device  and writes the config parameters.
\end{enumerate}

\item  \textbf{IRIG$\_$THREAD::run()}
\begin{enumerate}
	\item \textit{Inputs :} Nil
	\item \textit{Functional details:} This function is reads the time data from device
	\item \textit{Outputs:} Nil.
	\item \textit{Implementation details:} Qt Creator 5.4, C++ language.
	\item \textit{File operations:} Opens the irig device and reads the time data to UT1,UT2,CD1,CD2 arrays.
\end{enumerate}




%\end{enumerate}


\subsection{LogData}
\subsubsection{Attributes}
\begin{enumerate}
\item [$\blacklozenge$] QString TimFileName1,TimFileName2,TimFileName3;
\item [$\blacklozenge$] QString TimFileName4,TimFileName5;
	\item [$\blacklozenge$] int TimeSamplesCount1,TimeSamplesCount2,TimeSamplesCount3
   \item [$\blacklozenge$] int 	TimeSamplesCount4,TimeSamplesCount5;
	\item [$\blacklozenge$] quint8 AppExitFlag;
	\item [$\blacklozenge$] QFile  TimFile1;
	\item [$\blacklozenge$] qint64 TimFileSize1;
	\item [$\blacklozenge$] QFile  TimFile2;
	\item [$\blacklozenge$] qint64 TimFileSize2;
	\item [$\blacklozenge$] QFile  TimFile3;
	\item [$\blacklozenge$] qint64 TimFileSize3;
	\item [$\blacklozenge$] QFile  TimFile4;
	\item [$\blacklozenge$] qint64 TimFileSize4;
	\item [$\blacklozenge$] QFile  TimFile5;
	\item [$\blacklozenge$] qint64 TimFileSize5;
\end{enumerate}
\subsubsection{Member Functions }
\begin{enumerate}
	\item [$\blacklozenge$] void GetFileName(QString Suffix);
	\item [$\blacklozenge$] void LogCode1Data(quint8 Data[]);
	\item [$\blacklozenge$] void LogCode2Data(quint8 Data[]);
	\item [$\blacklozenge$] void LogCode3Data(quint8 Data[]);
	\item [$\blacklozenge$] void LogCode4Data(quint8 Data[]);
	\item [$\blacklozenge$] void LogCode5Data(quint8 Data[]);
	\item [$\blacklozenge$] QString GetDataString(quint8 Data[], int UT$\_$CDT);
	
	
\end{enumerate}


\subsubsection{Slots}
\begin{enumerate}
	\item [$\blacklozenge$] int CopyTime1File();
	\item [$\blacklozenge$] int CopyTime2File();
	\item [$\blacklozenge$] int CopyTime3File();
	\item [$\blacklozenge$] int CopyTime4File();
	\item [$\blacklozenge$] int CopyTime5File();
	
\end{enumerate}

\subsubsection{Signals}
\begin{enumerate}
	\item [$\blacklozenge$] int TimCopyFlag1();
	\item [$\blacklozenge$] int TimCopyFlag2();
	\item [$\blacklozenge$] int TimCopyFlag3();
	\item [$\blacklozenge$] int TimCopyFlag4();
	
\end{enumerate}

\subsubsection{Internal design details }
\begin{enumerate}
	\item  \textbf{void GetFileName(QString Suffix)}
	\begin{enumerate}
		\item \textit{Inputs :} Suffix - current time in string format.
		\item \textit{Functional details:} This function reads current data time ,initilises the file name variables 
		\item \textit{Outputs:} Nil.
		\item \textit{Implementation details:} Qt Creator 5.4, C++ language.
		\item \textit{File operations:} Nil.
	\end{enumerate}
	
	\item  \textbf{QString LogData::GetDataString(quint8 Data[], int UT$\_$CDT)}
	\begin{enumerate}
		\item \textit{Inputs :} Time data in structured format.
		\item \textit{Functional details:} This function formats the time in struct to string  
		\item \textit{Outputs:} Nil.
		\item \textit{Implementation details:} Qt Creator 5.4, C++ language.
		\item \textit{File operations:} Nil.
	\end{enumerate}
	

	
	\item  \textbf{void LogData::LogCode1Data(quint8 Data[])}
	\begin{enumerate}
		\item \textit{Inputs :} Time data .
		\item \textit{Functional details:} This function insert the UT-1time data in array which will be written to disk.
		\item \textit{Outputs:} Nil.
		\item \textit{Implementation details:} Qt Creator 5.4, C++ language.
		\item \textit{File operations:} Nil.
	\end{enumerate}
	
	\item  \textbf{void LogData::LogCode2Data(quint8 Data[])}
	\begin{enumerate}
		\item \textit{Inputs :} Time data .
		\item \textit{Functional details:} This function insert the UT-2 time data in array which will be written to disk.
		\item \textit{Outputs:} Nil.
		\item \textit{Implementation details:} Qt Creator 5.4, C++ language.
		\item \textit{File operations:} Nil.
	\end{enumerate}
	
	\item  \textbf{void LogData::LogCode3Data(quint8 Data[])}
	\begin{enumerate}
		\item \textit{Inputs :} Time data .
		\item \textit{Functional details:} This function insert the CDT-1 time data in array which will be written to disk.
		\item \textit{Outputs:} Nil.
		\item \textit{Implementation details:} Qt Creator 5.4, C++ language.
		\item \textit{File operations:} Nil.
	\end{enumerate}
	
	\item  \textbf{void LogData::LogCode4Data(quint8 Data[])}
	\begin{enumerate}
		\item \textit{Inputs :} Time data .
		\item \textit{Functional details:} This function insert the CDT-2 time data in array which will be written to disk.
		\item \textit{Outputs:} Nil.
		\item \textit{Implementation details:} Qt Creator 5.4, C++ language.
		\item \textit{File operations:} Nil.
	\end{enumerate}
	
	\item  \textbf{void LogData::LogCode5Data(quint8 Data[])}
	\begin{enumerate}
		\item \textit{Inputs :} Time data .
		\item \textit{Functional details:} This function insert the time data in array which will be written to disk.
		\item \textit{Outputs:} Nil.
		\item \textit{Implementation details:} Qt Creator 5.4, C++ language.
		\item \textit{File operations:} Nil.
	\end{enumerate}
	
	
	\item  \textbf{int LogData::CopyTime1File()}
	\begin{enumerate}
		\item \textit{Inputs :} Nil .
		\item \textit{Functional details:} This Slot function which will be called on emission of signal TimCopyFlag1,it will write the array of time samples to disk.
		\item \textit{Outputs:} Nil.
		\item \textit{Implementation details:} Qt Creator 5.4, C++ language.
		\item \textit{File operations:} Nil.
	\end{enumerate}
	
	\item  \textbf{int LogData::CopyTime2File()}
	\begin{enumerate}
		\item \textit{Inputs :} Nil .
		\item \textit{Functional details:} This Slot function which will be called on emission of signal TimCopyFlag1.it will write the array of time samples to disk.
		\item \textit{Outputs:} Nil.
		\item \textit{Implementation details:} Qt Creator 5.4, C++ language.
		\item \textit{File operations:} Nil.
	\end{enumerate}
	
	\item  \textbf{int LogData::CopyTime3File()}
	\begin{enumerate}
		\item \textit{Inputs :} Nil .
		\item \textit{Functional details:} This Slot function which will be called on emission of signal TimCopyFlag1,it will write the array of time samples to disk.
		\item \textit{Outputs:} Nil.
		\item \textit{Implementation details:} Qt Creator 5.4, C++ language.
		\item \textit{File operations:} Nil.
	\end{enumerate}
	\item  \textbf{int LogData::CopyTime4File()}
	\begin{enumerate}
		\item \textit{Inputs :} Nil .
		\item \textit{Functional details:} This Slot function which will be called on emission of signal TimCopyFlag1.it will write the array of time samples to disk.
		\item \textit{Outputs:} Nil.
		\item \textit{Implementation details:} Qt Creator 5.4, C++ language.
		\item \textit{File operations:} Nil.
	\end{enumerate}
	
\end{enumerate}


\subsection{NetworkTx}


\subsubsection{Attributes}
\begin{enumerate}
	\item [$\rhd$] QByteArray datagram;;
	\item [$\rhd$] int socketEnable;
	\item [$\rhd$] int Socket;
	
	\item [$\rhd$] QHostAddress ForeignHost;
	
	
	\item [$\rhd$] int ForeignPort;
	
	\item [$\rhd$] unsigned long int MCCserTxCount;
	\item [$\rhd$] unsigned long int MCCserErrCount;
	\item [$\rhd$] quint8 SerTxFlag;
	\item [$\rhd$] int NoOf$\_$TxSockets;
	
	
\end{enumerate}
\subsubsection{Member Functions }
\begin{enumerate}
	\item [$\blacklozenge$] int NetworkTxInit(int socketNo, QList$<$TXSOCKETNODE*$>$ net$\_$TxList, int NoOfTxSockets);
\end{enumerate}

\subsubsection{Slots}
\begin{enumerate}
	\item [$\blacklozenge$] void sendDatagram(QByteArray datagram);
\end{enumerate}
\subsubsection{Signals}
\begin{enumerate}
	\item [$\blacklozenge$] void netTxdFlag(QByteArray sup$\_$sys$\_$data);
\end{enumerate}

\subsubsection{Internal design details }
\begin{enumerate}
	\item  \textbf{ int NetworkTxInit(int socketNo, QList<TXSOCKETNODE*> net$\_$TxList, int NoOfTxSockets)}
	\begin{enumerate}
		\item \textit{Inputs :} Socket no, list of socket interfaces ,No of sockets
		\item \textit{Functional details:} This function intilises the network interface accoding to the config file "TxSocket.txt"
		\item \textit{Outputs:} Nil.
		\item \textit{Implementation details:} Nil.
		\item \textit{File operations:} Nil.
	\end{enumerate}

\item  \textbf{void sendDatagram(QByteArray datagram)}
\begin{enumerate}
	\item \textit{Inputs :} Time packet data in QByte array format.
	\item \textit{Functional details:} Sends the Ethernet packet 
	\item \textit{Outputs:} Nil.
	\item \textit{Implementation details:} Nil.
	\item \textit{File operations:} Nil.
\end{enumerate}
\end{enumerate}


\subsection{Network Reception: NetworkRx} 

This class is derived from the QObject class. It is instantiated in the MainWindow class.
\subsubsection{Attributes}
\begin{enumerate}
	\item [$\rhd$] int NoOfRxSockets
	\item [$\rhd$] QList$<$RXSOCKETNODE*$>$ net$\_$RxList
	\item [$\rhd$] int messageNo
\end{enumerate}

\subsubsection{Member Functions }
\begin{enumerate}
	\item [$\blacklozenge$] int ReadNetworkRxFile(QString file$\_$name)
\end{enumerate}


\subsubsection{Slots}
Nil

\subsubsection{Signals}
Nil


\subsubsection{Internal design details }
\begin{enumerate}
	\item  \textbf{NetworkRx::NetworkRx()}
	\begin{enumerate}
		\item \textit{Inputs :} Nil.
		\item \textit{Functional details:} The constructor is derived from the QThread class.
		\item \textit{Outputs:} Network Thread class generated.
		\item \textit{Implementation details:} Nil.
		\item \textit{File operations:} Nil.
	\end{enumerate}
	\item  \textbf{NetworkRx :: ReadNetworkRxFile(char* file$\_$name)}
	\begin{enumerate}
		\item \textit{Inputs :} Configuration file name.
		\item \textit{Functional details:} The configuration file containing the number of sockets and the network configuration parameters for each socket is opened and read. A linked list containing the configuration details of each socket is created.
		\item \textit{Outputs:} Socket configuration details read and stored. 
		\item \textit{Implementation details:} QT Linux 5.4 calls are used to implement this function.
		\item \textit{File operations:} UdpRxSocket.inp
	\end{enumerate}   
\end{enumerate}

\subsection{Network Transmission: NetworkTx}

\subsubsection{Attributes}
\begin{enumerate}
	\item [$\rhd$] int NoOfTxSockets
	\item [$\rhd$] QList$<$TXSOCKETNODE*$>$ net$\_$TxList
\end{enumerate}

\subsubsection{Member Functions }
\begin{enumerate}
	\item [$\blacklozenge$] int ReadNetworkTxFile(QString file$\_$name)
	\end{enumerate}


\subsubsection{Slots}
Nil

\subsubsection{Signals}
Nil

\subsubsection{Internal design details }
\begin{enumerate}
	\item  \textbf{NetworkRx :: ReadNetworkTxFile(QString file$\_$name)}
	\begin{enumerate}
		\item \textit{Inputs:} Configuration file name
		\item \textit{Functional details:} The configuration file containing the number of sockets and the network configuration parameters for each socket is opened and read. A linked list containing the configuration details of each socket is created.
		\item \textit{Outputs:} Socket configuration details read and stored. 
		\item \textit{Implementation details:} QT Linux 5.4 calls are used to implement this function.
		\item \textit{File operations:} UdpTxSocket.inp
	\end{enumerate}  
\end{enumerate}


\section{Threads detailed design}

\subsection{Network Reception Thread: NetworkRxThread} \label{NtWkRxThd}
\begin{enumerate}
	\item [$\rhd$] This is a worker thread and does not need any user interface.
	\item [$\rhd$] This thread starts running immediately after it is created and stops when the application is quit.
	\item [$\rhd$] It acquires UDP/IP packets in UNICAST mode i.e., receives CDM data.
	\item [$\rhd$] Data to be received from two sockets.
	\item [$\rhd$] The synchronization to the main thread is through signals.
\end{enumerate}
This class is derived from the QThread class. It is instantiated and started in the MainWindow class.
\subsubsection{Attributes}
\begin{enumerate}
	\item [$\rhd$] QUdpSocket *udpRxSocket
	\item [$\rhd$] int messageNo
	\item [$\rhd$] QByteArray RxSocketData
	\item [$\rhd$] bool NetworkRxInit(int,QList$<$RXSOCKETNODE*$>$)
	\item [$\rhd$] quint8 rxsocketno
	\item [$\rhd$] bool NetRx$\_$OK$\_$flag
	\item [$\rhd$] quint64 nopacketreceptioncount$\_$L1;
	\item [$\rhd$] quint64 nopacketreceptioncount$\_$L2;
	\item [$\rhd$] uint Line1packetSize;
	\item [$\rhd$] uint Line2packetSize;
\end{enumerate}

\subsubsection{Member Functions }
\begin{enumerate}
	\item [$\blacklozenge$] bool NetworkRxInit(int,QList $<$RXSOCKETNODE*$>$)
\end{enumerate}


\subsubsection{Slots}
\begin{enumerate}
	\item [$\blacklozenge$] void readPendingDatagrams()
	\item [$\blacklozenge$] void closesocket()
\end{enumerate}

\subsubsection{Signals}
\begin{enumerate}
	\item [$\blacklozenge$] void Line1CDMDataRxd(int,QByteArray)
	\item [$\blacklozenge$] void Line2CDMDataRxd(int,QByteArray)
\end{enumerate}


\subsubsection{Internal design details }
\begin{enumerate}
	\item  \textbf{NetworkRxThread::NetworkRxThread()}
	\begin{enumerate}
		\item \textit{Inputs :} Nil.
		\item \textit{Functional details:} The constructor is derived from the QThread class.
		\item \textit{Outputs:} Network Thread class generated.
		\item \textit{Implementation details:} Nil.
		\item \textit{File operations:} Nil.
	\end{enumerate}
	
	\item  \textbf{NetworkRxThread :: NetworkRxInit()}
	\begin{enumerate}
		\item \textit{Inputs :} Nil.
		\item \textit{Functional details:} The required number of UDP sockets are created and configured as per the details read and stored. A slot is identified to service the readyread signal generated on reception of data.
		\item \textit{Outputs:} Sockets are configured. 
		\item \textit{Implementation details:} QT Linux 5.4 calls are used to implement this function.
		\item \textit{File operations:} Nil.
	\end{enumerate}
	\item \textbf{NetworkRxThread :: readPendingDatagrams()}
	\begin{enumerate}
		\item \textit{Inputs :} Readyread signal.
		\item \textit{Functional details:} On the occurrence of readyread signal, the socket which raised the signal is identified and the corresponding data is copied to a QByteArray variable. A signal is then raised to the Mainwindow indicating the reception of new CDM data frame along with the socket number (1 or 2).
		\item \textit{Outputs:} CDM data is received.
		\item \textit{Implementation details:} QT Linux 5.4 calls are used to implement this function.
		\item \textit{File operations:} Nil.
	\end{enumerate}   
\end{enumerate}





\subsection{Network Transmission Threads: NetworkTx }

\subsubsection{Attributes}
\begin{enumerate}
	\item [$\rhd$] NetrxMc$\_$Txn datagram
	\item [$\rhd$] QByteArray mcc$\_$Tx$\_$Buf2
	\item [$\rhd$] QByteArray mcc$\_$Tx$\_$Buf1
	\item [$\rhd$] quint64 line2$\_$MCC$\_$Cnt
	\item [$\rhd$] quint64 line1$\_$MCC$\_$Cnt	
	\item [$\rhd$] QUdpSocket *udpTxSocket
	\item [$\rhd$] int messageNo
	\item [$\rhd$] QByteArray RxSocketData
	\item [$\rhd$] bool NetTx$\_$OK$\_$flag
	\item [$\rhd$] quint64 line1$\_$No$\_$Tx$\_$Count
	\item [$\rhd$] quint64 line2$\_$No$\_$Tx$\_$Count	
\end{enumerate}

\subsubsection{Member Functions }
\begin{enumerate}
	\item [$\blacklozenge$] void NetworkTxInit()
\end{enumerate}


\subsubsection{Slots}
\begin{enumerate}
	\item [$\blacklozenge$] void sendDatagram1(QByteArray ,int,quint8)
	\item [$\blacklozenge$] void sendDatagram2(QByteArray ,int,quint8)
\end{enumerate}

\subsubsection{Signals}
\begin{enumerate}
	\item [$\blacklozenge$] void SignalForTx2MCC(QByteArray ,int,quint8 )
\end{enumerate}

\subsubsection{Internal design details }
\begin{enumerate}
	\item  \textbf{NetworkRx :: NetworkTxInit()}
	\begin{enumerate}
		\item \textit{Inputs:} Nil.
		\item \textit{Functional details:} Nil.
		\item \textit{Outputs:} Nil. 
		\item \textit{Implementation details:} Nil.
		\item \textit{File operations:} Nil.
	\end{enumerate}
	\item \textbf{NetworkRx :: sendDatagrams1(QByteArray mcc$\_$buf$\_$Txn,  int i, quint8 Tx$\_$Packetsize)}
	\begin{enumerate}
		\item \textit{Inputs :} Byte array to be transmitted, line number, packet size to be transmitted, 
		\item \textit{Functional details:}  writeDatagram writes data on to the socket-1 and returns number of bytes transmitted if successful, else -1 will be returned. 
		\item \textit{Outputs:} Writes data on to the socket-1 for transmission.
		\item \textit{Implementation details:} QT Linux 5.4 calls are used to implement this function.
		\item \textit{File operations:} Nil.
	\end{enumerate}  
	\item \textbf{NetworkRx :: sendDatagrams2(QByteArray mcc$\_$buf$\_$Txn,  int i, quint8 Tx$\_$Packetsize)}
	\begin{enumerate}
	\item \textit{Inputs :} Byte array to be transmitted, line number, packet size to be transmitted, 
	\item \textit{Functional details:}  writeDatagram writes data on to the socket-2 and returns number of bytes transmitted if successful, else -1 will be returned. 
	\item \textit{Outputs:} Writes data on to the socket-2 for transmission.
	\item \textit{Implementation details:} QT Linux 5.4 calls are used to implement this function.
	\item \textit{File operations:} Nil.
	\end{enumerate}  
\end{enumerate}


%************************************
\subsection{siothreads: SioThread, SioThreadE}



\subsubsection{Attributes}

\textbf{SioThread : For CDT and EDAU}\\
\begin{enumerate}
	\item [$\rhd$] QSerialPort *serial	
	\item [$\rhd$] QByteArray sio$\_$data
	\item [$\rhd$] quint8 SIO$\_$Status
	\item [$\rhd$] qint64 Total$\_$Bytes
	\item [$\rhd$] quint8 sio1$\_$ret
	\item [$\rhd$] quint8 sioPort
	\item [$\rhd$] bool serialclear
	\item [$\rhd$] bool sio$\_$OK$\_$ststus$\_$flag
	\item [$\rhd$] quint64 data1normalcount
	\item [$\rhd$] quint64 data2normalcount
	\item [$\rhd$] quint64 data1absentcount
	\item [$\rhd$] quint64 data2absentcount
	\item [$\rhd$] uint Rxd1Bytes
	\item [$\rhd$] uint Rxd2Bytes
\end{enumerate}

\subsubsection{Member Functions } 
\begin{enumerate}
	\item [$\rhd$] Sio()
	\item [$\rhd$] int Read$\_$Sio$\_$File(QString file$\_$name)
\end{enumerate}
\textbf{SioThread : For CDT and EDAU}\\
\begin{enumerate}
	\item [$\blacklozenge$] int ConfigureSioCard(int i, QList$<$SIO$\_$Node*$>$ sio$\_$list, qint64 TotalBytes)
\end{enumerate}
\subsubsection{Slots}
\textbf{SioThread : For CDT and EDAU}\\
\begin{enumerate}
	\item [$\blacklozenge$] void readData1()
\end{enumerate}
\begin{enumerate}
	\item [$\blacklozenge$] void readData2()
\end{enumerate}

\subsubsection{Signals}
\textbf{SioThread : For CDT and EDAU}\\
\begin{enumerate}
	\item [$\blacklozenge$] void CDT1$\_$recvd(int,QByteArray)
	\item [$\blacklozenge$]void CDT2$\_$recvd(int,QByteArray)
	\item [$\blacklozenge$] void EDAU1$\_$recvd(int,QByteArray)
	\item [$\blacklozenge$]void EDAU2$\_$recvd(int,QByteArray)
\end{enumerate}

\subsubsection{Internal design details }
\begin{enumerate}
	\item  \textbf{Sio :: Sio()}
	\begin{enumerate}
		\item \textit{Inputs :} Nil.
		\item \textit{Functional details:} Objects of type QSerialPort are instantiated for 8 serial ports.
		\item \textit{Outputs:} 8 serial port objects are created.
		\item \textit{Implementation details:} 	QT Linux 5.4 libraries are used to implement the above.
		\item \textit{File operations:} Nil.
	\end{enumerate}
	
	\item  \textbf{Sio :: Read$\_$Sio$\_$File(QString file$\_$name)}
	\begin{enumerate}
		\item \textit{Inputs :} Configuration input file name.
		\item \textit{Functional details:} The number of serial ports and the corresponding configuration details are read and a linked list of serial port details is generated. 
		\item \textit{Outputs:} Serial port configuration details read and stored.
		\item \textit{Implementation details:} QT Linux 5.4 libraries are used to implement the above.
		\item \textit{File operations:} sio.txt.
	\end{enumerate}
	
	\item  \textbf{SioThread :: int ConfigureSioCard(int i, QList$<$SIO$\_$Node*$>$ sio$\_$list, qint64 TotalBytes)}
	\begin{enumerate}
		\item \textit{Inputs :} Serial port number, total bytes, serial linked list.
		\item \textit{Functional details:} The required number of serial ports are as per the details stored in the linked list. 
		\item \textit{Outputs:} Serial ports configured. 
		\item \textit{Implementation details:} QT Linux 5.4 libraries are used to implement the above.
		\item \textit{File operations:} Nil.
	\end{enumerate}

\end{enumerate}

\section{Data detailed design}

\subsection{Data structures}
\subsubsection{CDT}
\begin{enumerate}
	\item quint8 units$\_$10Hz 
	\item quint8 sign       
	\item quint8 lift$\_$off  
	\item quint8 hmsec    
	\item quint8 sec       
	\item quint8 min        
	\item quint8 hrs        
	\item quint8 hold
\end{enumerate}
\subsubsection{Constants}
\begin{enumerate}
	\item int tc$\_$chain;   
	\item float azenc$\_$wild$\_$sample$\_$tolerance;
	\item float elenc$\_$wild$\_$sample$\_$tolerance;
	\item float enc1$\_$az$\_$bias;
	\item float enc1$\_$el$\_$bias;
	\item float enc2$\_$az$\_$bias;
	\item float enc2$\_$el$\_$bias;
	\item float az$\_$cdm$\_$slope;
	\item float az$\_$prog$\_$slope;
	\item float el$\_$cdm$\_$slope;
	\item float el$\_$prog$\_$slope;
	\item int analog$\_$channel$\_$no; 
	\item float PA$\_$Upperlimit;
	\item float PA$\_$lowerlimit;
	\item QString prog$\_$file1;
	\item uint DIO$\_$PortNo$\_$ServoOpModes;
	\item uint DIO$\_$PortNo$\_$ServoDrive$\_$RCU3Bits;
	\item uint DIO$\_$PortNo$\_$DCERCUStatus;
	\item uint DIO$\_$PortNo$\_$TxBiasSys;
	\item uint DIO$\_$PortNo$\_$CRIPBLRHealth;
	\item uint DIO$\_$PortNo$\_$CRISHARHealth;
	\item uint DIO$\_$NoOf$\_$Ports;
	\item float Az$\_$UpperLimit;
	\item float Az$\_$LowerLimit;
	\item float El$\_$UpperLimit;
	\item float El$\_$lowerLimit;
	\item int Dio$\_$Devicenumber;
	\item qint32 Dio$\_$Item;
	\item qint32 Ai$\_$Devicenumber;
	\item qint32 An$\_$Item;
	\item quint8 LogCnterDelay;
	\item QString logfolder;	
\end{enumerate}

\subsubsection{AntLog}
\begin{enumerate}
	\item quint64 ProcessCount
	\item qint64 Line1$\_$EHMS
	\item qint64 Line2$\_$EHMS
	\item qint64 Selected$\_$EHMS
	\item int CDT$\_$SelectedLine
	\item char ext$\_$int
	\item uint Az$\_$DriveOn$\_$TcProcess
	\item uint El$\_$DriveOn$\_$TcProcess
	\item uint EmStop$\_$TcProcess
	\item uint Az$\_$OpMode$\_$TcProcess
	\item uint El$\_$OpMode$\_$TcProcess
	\item float Commanded$\_$Az$\_$El$\_$TcProcess[2]
	\item float AntennaPosition$\_$Az$\_$El$\_$TcProcess[2]
	\item float Error$\_$Az$\_$El$\_$TcProcess[2]
	\item float Az$\_$ch1
	\item float El$\_$ch1
	\item float Az$\_$ch2
	\item float El$\_$ch2
	\item bool EDAUCurrentChainSelectedFlag
	\item bool EDAUCh1$\_$OverallHealth
	\item bool EDAUCh2$\_$OverallHealth
	\item bool EDAUCh1$\_$IntExtStatusforProc
	\item bool EDAUCh2$\_$IntExtStatusforProc
	\item int CDM$\_$SelectedLine
	\item int CDM$\_$SelectedSource
	\item float Az$\_$CDM1$\_$Src1
	\item float El$\_$CDM1$\_$Src1
	\item float Az$\_$CDM1$\_$Src2
	\item float El$\_$CDM1$\_$Src2
	\item float Az$\_$CDM2$\_$Src1
	\item float El$\_$CDM2$\_$Src1
	\item float Az$\_$CDM2$\_$Src2
	\item float El$\_$CDM2$\_$Src2
	\item uint ID$\_$CDM1$\_$Src1
	\item uint ID$\_$CDM1$\_$Src2
	\item uint ID$\_$CDM2$\_$Src1
	\item uint ID$\_$CDM2$\_$Src2
	\item uint CompFlag$\_$CDM1
	\item uint CompFlag$\_$CDM2
	\item quint64 cdm1$\_$RxCount
	\item quint64 cdm2$\_$RxCount
	\item quint64 Pres$\_$cdm$\_$L1$\_$Src1$\_$Count
	\item quint64 Pres$\_$cdm$\_$L1$\_$Src2$\_$Count
	\item quint64 Pres$\_$cdm$\_$L2$\_$Src1$\_$Count
	\item quint64 Pres$\_$cdm$\_$L2$\_$Src2$\_$Count
	\item quint64 line1$\_$MCC$\_$Cnt
	\item quint64 line2$\_$MCC$\_$Cnt
	\item bool Antlog$\_$Selection$\_$Flag
	\item quint64 AntlogCounter
\end{enumerate}

\subsubsection{Acquired$\_$encoder$\_$Data}
\begin{enumerate}
	\item qint64 Line1$\_$EHMS
	\item qint64 Line2$\_$EHMS
	\item qint64 Selected$\_$EHMS
	\item int EDAUChain1$\_$id
	\item float Az$\_$hnds$\_$ch1
	\item float Az$\_$tens$\_$ch1
	\item float Az$\_$ones$\_$ch1
	\item float Az$\_$1tenth$\_$ch1
	\item float Az$\_$1hundredth$\_$ch1
	\item float El$\_$hnds$\_$ch1
	\item float El$\_$tens$\_$ch1
	\item float El$\_$ones$\_$ch1
	\item float El$\_$1tenth$\_$ch1
	\item float El$\_$1hundredth$\_$ch1
	\item char Edau1$\_$ExtIntLog
	\item int EDAUchain1$\_$HealthByte
	\item int EDAUChain1$\_$EndByte
	\item float Az$\_$ch1
	\item float El$\_$ch1
	\item int EDAUChain2$\_$id
	\item float Az$\_$hnds$\_$ch2
	\item float Az$\_$tens$\_$ch2
	\item float Az$\_$ones$\_$ch2
	\item float Az$\_$1tenth$\_$ch2
	\item float Az$\_$1hundredth$\_$ch2
	\item float El$\_$hnds$\_$ch2
	\item float El$\_$tens$\_$ch2
	\item float El$\_$ones$\_$ch2
	\item float El$\_$1tenth$\_$ch2
	\item float El$\_$1hundredth$\_$ch2
	\item char Edau2$\_$ExtIntLog
	\item int EDAUchain2$\_$HealthByte
	\item int EDAUChain2$\_$EndByte
	\item float Az$\_$ch2
	\item float El$\_$ch2
	\item bool EDAUCh1$\_$OverallHealth
	\item bool EDAUCh2$\_$OverallHealth
	\item quint32 EDAUCh1$\_$NotHelathCnt
	\item quint32 EDAUCh2$\_$NotHelathCnt
	\item quint32 EDAUCh1$\_$Data$\_$NotProcessedCnt
	\item quint32 EDAUCh2$\_$Data$\_$NotProcessedCnt
	\item bool EDAUCh1$\_$IntExtStatusforProc
	\item bool EDAUCh2$\_$IntExtStatusforProc
	\item bool EDAUCurrentChainSelectedFlag
	\item float EDAUDataforProc$\_$Az
	\item float EDAUDataforProc$\_$El
	\item bool EDAUCh1 InitialisedFlag
	\item bool EDAUCh2 InitialisedFlag
	\item bool EDAUCh1 JumpsFlag
	\item bool EDAUCh2 JumpsFlag
	\item quint64 EDAUCh1 WildSampleCnt
	\item quint64 EDAUCh2 WildSampleCnt
\end{enumerate}

\subsubsection{NetRxMc}
\begin{enumerate}
	\item quint32 range1
	\item quint32   azimuth1
	\item quint32 elevation1 
	\item quint32 radartime
	\item quint32 cdm$\_$source1
	\item quint32 range2 
	\item quint32 azimuth2
	\item quint32 elevation2 
	\item quint32 rangetime
	\item quint32 cdm$\_$source2 
	\item quint32 sys$\_$id
	\item quint32 comp$\_$flags
\end{enumerate}

\subsubsection{Stnlog}
\begin{enumerate}
	\item uint station$\_$Id
	\item uint system$\_$In$\_$use
	\item quint64 ProcessCount
	\item qint64 Line1$\_$EHMS
	\item qint64 Line2$\_$EHMS
	\item qint64 Selected$\_$EHMS
	\item char ext$\_$int
	\item bool TcReadyNotReady$\_$TcProcess
	\item bool TcRadiateNotRadiate$\_$TcProcess
	\item uint PowerStatus
	\item float powerLevel
	\item bool DCE$\_$RCU$\_$Health$\_$TcProcess
	\item bool DCE$\_$RCU$\_$ComdControl$\_$TcProcess
	\item uint DCE$\_$RCU$\_$ComdWord$\_$TcProcess
	\item quint64 cdm1$\_$RxCount
	\item quint64 cdm2$\_$RxCount
	\item quint64 Pres$\_$cdm$\_$L1$\_$Src1$\_$Count
	\item quint64 Pres$\_$cdm$\_$L1$\_$Src2$\_$Count
	\item quint64 Pres$\_$cdm$\_$L2$\_$Src1$\_$Count
	\item quint64 Pres$\_$cdm$\_$L2$\_$Src2$\_$Count
	\item quint64 line1$\_$MCC$\_$Cnt
	\item quint64 line2$\_$MCC$\_$Cnt
	\item quint8 CRI$\_$SHAR$\_$Health$\_$Processor$\_$TcProcess
	\item quint8 CRI$\_$SHAR$\_$Health$\_$Eth1$\_$TcProcess
	\item quint8 CRI$\_$SHAR$\_$Health$\_$Eth2$\_$TcProcess
	\item quint8 CRI$\_$SHAR$\_$Health$\_$DI$\_$TcProcess
	\item quint8 CRI$\_$SHAR$\_$Health$\_$DO$\_$TcProcess
	\item quint8 CRI$\_$SHAR$\_$Health$\_$RUN$\_$TcProcess
	\item bool CRI$\_$PBLR$\_$Health$\_$Processor$\_$TcProcess
	\item bool CRI$\_$PBLR$\_$Health$\_$Eth1$\_$TcProcess
	\item bool CRI$\_$PBLR$\_$Health$\_$Eth2$\_$TcProcess
	\item bool CRI$\_$PBLR$\_$Health$\_$DI$\_$TcProcess
	\item bool CRI$\_$PBLR$\_$Health$\_$DO$\_$TcProcess
	\item bool CRI$\_$PBLR$\_$Health$\_$RUN$\_$TcProcess
	\item quint8 RCU$\_$3Bit$\_$Status$\_$TcProcess
	\item bool Stnlog$\_$Selection$\_$Flag
	\item bool Dio$\_$OkStatus$\_$flag
	\item quint64 AO$\_$WriteSuccessCount
	\item quint64 AO$\_$WriteErrorCount
	\item quint64 Sio1$\_$datanormalcount
	\item quint64 Sio1$\_$dataabsentcount
	\item quint64 Sio2$\_$datanormalcount
	\item quint64 Sio2$\_$dataabsentcount
	\item quint64 Sio3$\_$datanormalcount
	\item quint64 Sio3$\_$dataabsentcount
	\item quint64 Sio4$\_$datanormalcount
	\item quint64 Sio4$\_$dataabsentcount
	\item quint64 nopacketRxCount$\_$L1
	\item quint64 nopacketRxCount$\_$L2
	\item quint64 StnlogCounter
\end{enumerate}

\subsubsection{DCE$\_$RCUlog}
\begin{enumerate}
	\item char Ch1$\_$sign      
	\item quint8 Ch1$\_$hrs        
	\item quint8 Ch1$\_$min        
	\item quint8 Ch1$\_$sec      
	\item quint8 Ch1$\_$hmsec    
	\item char Ch2$\_$sign      
	\item quint8 Ch2$\_$hrs        
	\item quint8 Ch2$\_$min       
	\item quint8 Ch2$\_$sec       
	\item quint8 Ch2$\_$hmsec    
	\item quint8 DCE$\_$RCU$\_$ComdWord
	\item bool DCE$\_$RCU$\_$Health
	\item bool DCE$\_$RCU$\_$LocalRemote
	\item bool SynthesisedPowerHealth
\end{enumerate}

\subsubsection{NetTxMCC}
\begin{enumerate}
	\item quint8 stationid
	\item quint8 chainid
	\item quint8 tcchainsts
	\item quint8 comdencsts
	\item quint8 txpwrsts
	\item quint8 servosts
	\item quint8 azcdmlsb
	\item quint8 azcdmmsb
	\item quint8 azposlsb
	\item quint8 azposmsb
	\item quint8 azerrlsb
	\item quint8 azerrmsb
	\item quint8 elcdmlsb
	\item quint8 elcdmmsb
	\item quint8  elposlsb
	\item quint8  elposmsb
	\item quint8 encochaininfo
	\item quint8 cdtsec
	\item quint8 cdtmin
	\item quint8 cdthrs
	\item quint8 cdt$\_$hmsec
	\item quint8 elerrlsb
	\item quint8 elerrmsb
	\item quint8 block$\_$id$\_$msb
	\item quint8 block$\_$id$\_$lsb
	\item quint8 tx$\_$PLC$\_$Status$\_$SHAR
	\item quint8 tx$\_$PLC$\_$Status$\_$PBLR
	\item quint8 tx$\_$PLC$\_$Status$\_$SHAR$\_$RCU$\_$3Bit
	\item quint64 Mcc$\_$Txline$\_$count
                  
\end{enumerate}

\subsubsection{cdm1$\_$buffer}
\begin{enumerate}
	\item quint8quint64 ProcessCount
\item qint64 Line1$\_$EHMS
\item qint64 Line2$\_$EHMS
\item qint64 Selected$\_$EHMS
\item char ext$\_$int
\item int range1
\item int azimuth1
\item int elevation1
\item int Radartime
\item int cdm$\_$source1
\item int range2
\item int azimuth2
\item int elevation2
\item int Rangetime
\item int cdm$\_$source2
\item int sys$\_$id
\item int comp$\_$flag
\item quint8 packetsize
\item quint64 cdm1$\_$RxCount
\item quint64 Pres$\_$cdm$\_$L1$\_$Src1$\_$Count
\item quint64 Pres$\_$cdm$\_$L1$\_$Src2$\_$Count
\item quint64 Cdm1logCounter
\end{enumerate}

\subsubsection{cdm2$\_$buffer}
\begin{enumerate}
	\item quint8quint64 ProcessCount
	\item qint64 Line1$\_$EHMS
	\item qint64 Line2$\_$EHMS
	\item qint64 Selected$\_$EHMS
	\item char ext$\_$int
	\item int range1
	\item int azimuth1
	\item int elevation1
	\item int Radartime
	\item int cdm$\_$source1
	\item int range2
	\item int azimuth2
	\item int elevation2
	\item int Rangetime
	\item int cdm$\_$source2
	\item int sys$\_$id
	\item int comp$\_$flag
	\item quint8 packetsize
	\item quint64 cdm2$\_$RxCount
	\item quint64 Pres$\_$cdm$\_$L2$\_$Src1$\_$Count
	\item quint64 Pres$\_$cdm$\_$L2$\_$Src2$\_$Count
	\item quint64 Cdm2logCounter
\end{enumerate}

\subsection{Input data files}

The details of input files are given in section. \ref{Section:InpFileFormat} and the corresponding pin/port assignments along with formats are given in section. \ref{Section:ConnectDataPortsFormat}.

\subsection{Configuration files}
The details of input configuration files are given in section. \ref{Section:InpFileFormat} and the corresponding pin/port assignments along with formats are given in section. \ref{Section:ConnectDataPortsFormat}.

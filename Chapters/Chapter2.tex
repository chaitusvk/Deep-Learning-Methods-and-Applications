\chapter{Decomposition description}
\label{Chapter2}
\section{Module decomposition}
\subsection{main}
\begin{enumerate}
	\item [$\blacklozenge$] To create an object for this application.
	\item [$\blacklozenge$] To initialize the settings instance for this application.	
\end{enumerate}
 
  \subsection{mainwindow}
  
  \begin{enumerate}
  	\item [$\blacklozenge$] To create the objects for the process related  classes.
  	\item [$\blacklozenge$] To read the configuration information from the input files and Network transmission.
  	\item [$\blacklozenge$] To create and begin the threads for serial i/o threads (4 nos), network reception threads (2 nos), network transmission threads (2 nos), DIO parallel (1 nos), analog input (1 nos), analog output (1 nos) and logging (1 nos).
  	\item [$\blacklozenge$] To facilitate the selection and updation of various toolbar buttons provided for user selectable functions.
  	\item [$\blacklozenge$] To start a 100 msec timer to process the acquired data periodically.
  	\item [$\blacklozenge$] To map the signals to the corresponding slots used in this class.
  	\item [$\blacklozenge$] To update the parameters for logging and parameters for plotting.
  	\item [$\blacklozenge$] To frame the data for transmission to MCC
  	\item [$\blacklozenge$] To provide display updation and graphic user interface
  		
  \end{enumerate}

    \subsection{IrigRead}
    \begin{enumerate}
    	\item [$\blacklozenge$] To read the config file "irig.txt".
    	\item [$\blacklozenge$] To Initialize the PCI variables like No.of IRIG cards ,Interrupt Rate,etc .
    \end{enumerate}
    
\section{Concurrent process decomposition}

Nil. There is only one process running at a time in this project. But there are multiple threads running. The details are as follows.
       \subsection{IrigThread} 
       
        \begin{enumerate}
       	\item [$\blacklozenge$] This is a worker thread and does not need any user interface. 
       	\item [$\blacklozenge$] This thread starts running immediately after it is created and stops only when the application is quit.  
       	\item [$\blacklozenge$] It acquires Time Data from 4 IRIG-PCI cards.
       	\item [$\blacklozenge$] The synchronization to the main thread is through signals.
       	
       \end{enumerate}	
       
       \subsection{networkTxThread} 
   
       \begin{enumerate}
    \item [$\blacklozenge$] This is a worker thread and does not need any user interface.
    \item [$\blacklozenge$] This thread starts running immediately after it is created and stops only when the application is quit.
    \item [$\blacklozenge$] This Thread Initilises the ips address of the system
   	\item [$\blacklozenge$] This Thread send the Data through Ethernet port
   
   \end{enumerate}
	
   \subsection{LogData}
      \begin{enumerate}
   	\item [$\blacklozenge$] Logs the Time data at specified rates to hard disk.
   \end{enumerate}

\section{Data decomposition}

The data entities used in this project are grouped into the following categories:
       \subsection{Input data files} 
       
        \begin{enumerate}
        \item [$\blacklozenge$] irig.txt file contains parameters for the IRIG-PCI cards. The corresponding format of this input file is given in section. \ref{section:irig_card} 
       	
       	\item [$\blacklozenge$] Tx.txt file contains configuration Tx IPs . System Ip and corresponding Port numbers The corresponding format of this input file is given in section. \ref{section:tx.} 
       	
       \end{enumerate}	
	
       \subsection{Output data files} 
       
       \begin{enumerate}
       	\item [$\blacklozenge$] Log File of Time Data 
       	. \ref{Table:tim_Log_file}
       	
       \end{enumerate}		